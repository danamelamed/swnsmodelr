\documentclass[]{article}
\usepackage{lmodern}
\usepackage{amssymb,amsmath}
\usepackage{ifxetex,ifluatex}
\usepackage{fixltx2e} % provides \textsubscript
\ifnum 0\ifxetex 1\fi\ifluatex 1\fi=0 % if pdftex
  \usepackage[T1]{fontenc}
  \usepackage[utf8]{inputenc}
\else % if luatex or xelatex
  \ifxetex
    \usepackage{mathspec}
  \else
    \usepackage{fontspec}
  \fi
  \defaultfontfeatures{Ligatures=TeX,Scale=MatchLowercase}
\fi
% use upquote if available, for straight quotes in verbatim environments
\IfFileExists{upquote.sty}{\usepackage{upquote}}{}
% use microtype if available
\IfFileExists{microtype.sty}{%
\usepackage{microtype}
\UseMicrotypeSet[protrusion]{basicmath} % disable protrusion for tt fonts
}{}
\usepackage[margin=1in]{geometry}
\usepackage{hyperref}
\hypersetup{unicode=true,
            pdftitle={Model Selection},
            pdfborder={0 0 0},
            breaklinks=true}
\urlstyle{same}  % don't use monospace font for urls
\usepackage{color}
\usepackage{fancyvrb}
\newcommand{\VerbBar}{|}
\newcommand{\VERB}{\Verb[commandchars=\\\{\}]}
\DefineVerbatimEnvironment{Highlighting}{Verbatim}{commandchars=\\\{\}}
% Add ',fontsize=\small' for more characters per line
\usepackage{framed}
\definecolor{shadecolor}{RGB}{248,248,248}
\newenvironment{Shaded}{\begin{snugshade}}{\end{snugshade}}
\newcommand{\KeywordTok}[1]{\textcolor[rgb]{0.13,0.29,0.53}{\textbf{#1}}}
\newcommand{\DataTypeTok}[1]{\textcolor[rgb]{0.13,0.29,0.53}{#1}}
\newcommand{\DecValTok}[1]{\textcolor[rgb]{0.00,0.00,0.81}{#1}}
\newcommand{\BaseNTok}[1]{\textcolor[rgb]{0.00,0.00,0.81}{#1}}
\newcommand{\FloatTok}[1]{\textcolor[rgb]{0.00,0.00,0.81}{#1}}
\newcommand{\ConstantTok}[1]{\textcolor[rgb]{0.00,0.00,0.00}{#1}}
\newcommand{\CharTok}[1]{\textcolor[rgb]{0.31,0.60,0.02}{#1}}
\newcommand{\SpecialCharTok}[1]{\textcolor[rgb]{0.00,0.00,0.00}{#1}}
\newcommand{\StringTok}[1]{\textcolor[rgb]{0.31,0.60,0.02}{#1}}
\newcommand{\VerbatimStringTok}[1]{\textcolor[rgb]{0.31,0.60,0.02}{#1}}
\newcommand{\SpecialStringTok}[1]{\textcolor[rgb]{0.31,0.60,0.02}{#1}}
\newcommand{\ImportTok}[1]{#1}
\newcommand{\CommentTok}[1]{\textcolor[rgb]{0.56,0.35,0.01}{\textit{#1}}}
\newcommand{\DocumentationTok}[1]{\textcolor[rgb]{0.56,0.35,0.01}{\textbf{\textit{#1}}}}
\newcommand{\AnnotationTok}[1]{\textcolor[rgb]{0.56,0.35,0.01}{\textbf{\textit{#1}}}}
\newcommand{\CommentVarTok}[1]{\textcolor[rgb]{0.56,0.35,0.01}{\textbf{\textit{#1}}}}
\newcommand{\OtherTok}[1]{\textcolor[rgb]{0.56,0.35,0.01}{#1}}
\newcommand{\FunctionTok}[1]{\textcolor[rgb]{0.00,0.00,0.00}{#1}}
\newcommand{\VariableTok}[1]{\textcolor[rgb]{0.00,0.00,0.00}{#1}}
\newcommand{\ControlFlowTok}[1]{\textcolor[rgb]{0.13,0.29,0.53}{\textbf{#1}}}
\newcommand{\OperatorTok}[1]{\textcolor[rgb]{0.81,0.36,0.00}{\textbf{#1}}}
\newcommand{\BuiltInTok}[1]{#1}
\newcommand{\ExtensionTok}[1]{#1}
\newcommand{\PreprocessorTok}[1]{\textcolor[rgb]{0.56,0.35,0.01}{\textit{#1}}}
\newcommand{\AttributeTok}[1]{\textcolor[rgb]{0.77,0.63,0.00}{#1}}
\newcommand{\RegionMarkerTok}[1]{#1}
\newcommand{\InformationTok}[1]{\textcolor[rgb]{0.56,0.35,0.01}{\textbf{\textit{#1}}}}
\newcommand{\WarningTok}[1]{\textcolor[rgb]{0.56,0.35,0.01}{\textbf{\textit{#1}}}}
\newcommand{\AlertTok}[1]{\textcolor[rgb]{0.94,0.16,0.16}{#1}}
\newcommand{\ErrorTok}[1]{\textcolor[rgb]{0.64,0.00,0.00}{\textbf{#1}}}
\newcommand{\NormalTok}[1]{#1}
\usepackage{graphicx,grffile}
\makeatletter
\def\maxwidth{\ifdim\Gin@nat@width>\linewidth\linewidth\else\Gin@nat@width\fi}
\def\maxheight{\ifdim\Gin@nat@height>\textheight\textheight\else\Gin@nat@height\fi}
\makeatother
% Scale images if necessary, so that they will not overflow the page
% margins by default, and it is still possible to overwrite the defaults
% using explicit options in \includegraphics[width, height, ...]{}
\setkeys{Gin}{width=\maxwidth,height=\maxheight,keepaspectratio}
\IfFileExists{parskip.sty}{%
\usepackage{parskip}
}{% else
\setlength{\parindent}{0pt}
\setlength{\parskip}{6pt plus 2pt minus 1pt}
}
\setlength{\emergencystretch}{3em}  % prevent overfull lines
\providecommand{\tightlist}{%
  \setlength{\itemsep}{0pt}\setlength{\parskip}{0pt}}
\setcounter{secnumdepth}{0}
% Redefines (sub)paragraphs to behave more like sections
\ifx\paragraph\undefined\else
\let\oldparagraph\paragraph
\renewcommand{\paragraph}[1]{\oldparagraph{#1}\mbox{}}
\fi
\ifx\subparagraph\undefined\else
\let\oldsubparagraph\subparagraph
\renewcommand{\subparagraph}[1]{\oldsubparagraph{#1}\mbox{}}
\fi

%%% Use protect on footnotes to avoid problems with footnotes in titles
\let\rmarkdownfootnote\footnote%
\def\footnote{\protect\rmarkdownfootnote}

%%% Change title format to be more compact
\usepackage{titling}

% Create subtitle command for use in maketitle
\newcommand{\subtitle}[1]{
  \posttitle{
    \begin{center}\large#1\end{center}
    }
}

\setlength{\droptitle}{-2em}
  \title{Model Selection}
  \pretitle{\vspace{\droptitle}\centering\huge}
  \posttitle{\par}
  \author{}
  \preauthor{}\postauthor{}
  \date{}
  \predate{}\postdate{}


\begin{document}
\maketitle

Daily mean temperatures in Nova Scotia were interpolated based on models
selected from testing on a subset of temperatures from 2012. The amount
of records in one day was insufficient to build daily models, given the
quantity of covariates we wish to include. Generating models on a weekly
basis, with smooth terms that are dependent on the day of the year,
allows for all covariates to be used, and for daily variability to still
be addressed.\\
The models were generated over the study station data, plus data from
some external stations that were located at unrepresented areas of SWNS.
The model accuracies were validated by running them over external
stations, plus some study stations, since many external stations are in
very close proximity to study stations.

\subsubsection{Determining stations for modelling and
validating}\label{determining-stations-for-modelling-and-validating}

In the package \texttt{swnsmodelr}, temperature records from 98 stations
in SWNS for the years 2012 - 2018 are stored in the data frame
\texttt{swns\_stations\_df}. The station I.D.s for the study stations
and external stations are stored as lists in
\texttt{study\_stations\_list} and \texttt{external\_stations\_list},
respectively. The locations of these two lists of stations are plotted
in Figure 1 over a digital elevation model (DEM) of SWNS, after
filtering \texttt{swns\_stations\_df} by the two lists.

\begin{Shaded}
\begin{Highlighting}[]
\NormalTok{study_stations_df <-}\StringTok{ }\NormalTok{swns_stations_df }\OperatorTok\StringTok{ }\KeywordTok{filter}\NormalTok{(stationid }\OperatorTok\StringTok{ }\NormalTok{study_stations_list)}
\NormalTok{ext_stations_df  <-}\StringTok{ }\NormalTok{swns_stations_df }\OperatorTok\StringTok{ }\KeywordTok{filter}\NormalTok{(stationid }\OperatorTok\StringTok{ }\NormalTok{ext_stations_list)}

\NormalTok{study_stations_sp <-}\StringTok{ }\NormalTok{study_stations_df }\OperatorTok\StringTok{ }\KeywordTok{filter}\NormalTok{(}\OperatorTok{!}\KeywordTok{duplicated}\NormalTok{(stationid))}
\KeywordTok{coordinates}\NormalTok{(study_stations_sp) =}\StringTok{ }\ErrorTok{~}\StringTok{ }\NormalTok{EASTING  }\OperatorTok{+}\StringTok{ }\NormalTok{NORTHING}

\NormalTok{ext_stations_sp <-}\StringTok{ }\NormalTok{ext_stations_df }\OperatorTok\StringTok{ }\KeywordTok{filter}\NormalTok{(}\OperatorTok{!}\KeywordTok{duplicated}\NormalTok{(stationid))}
\KeywordTok{coordinates}\NormalTok{(ext_stations_sp) =}\StringTok{ }\ErrorTok{~}\StringTok{ }\NormalTok{EASTING  }\OperatorTok{+}\StringTok{ }\NormalTok{NORTHING}

\KeywordTok{plot}\NormalTok{(rasters_list[[}\DecValTok{1}\NormalTok{]])}
\KeywordTok{points}\NormalTok{(study_stations_sp, }\DataTypeTok{pch =} \DecValTok{0}\NormalTok{)}
\KeywordTok{points}\NormalTok{(ext_stations_sp, }\DataTypeTok{pch =} \DecValTok{5}\NormalTok{)}
\end{Highlighting}
\end{Shaded}

\begin{figure}
\centering
\includegraphics{model_selection_vignette_files/figure-latex/unnamed-chunk-49-1.pdf}
\caption{\textbf{Figure 1}: Locations of study stations (square, n = 72)
and external stations (diamond, n = 27) in Southwest Nova Scotia.}
\end{figure}

The study stations alone do not provide sufficient coverage of SWNS in
the eastern most portion surrouding the Halifax area, and the Annapolis
valley. There are external locations in those areas which were included
in the model stations along with the study stations.

\begin{itemize}
\tightlist
\item
  The region around Halifax doesn't have coverage of study stations.
\item
  Many external stations are basically overlapping study stations, which
  would give biased estimate of accuracy.
\item
  Therefore, model stations will be study stations but also include some
  external stations.
\item
  Validation stations will be external stations and some study stations
  that are a good distance from model stations.
\item
  The data frame swns\_stations\_df\_200 has temperature data from all
  stations plus environmental variables from 200m resolution.
\item
  It will be divided into the model stations and validation stations.
\end{itemize}

\begin{Shaded}
\begin{Highlighting}[]
\CommentTok{# MODEL STATIONS}
\NormalTok{swns_stations_df_}\DecValTok{200}\NormalTok{ <-}\StringTok{ }\KeywordTok{add_date_columns}\NormalTok{(swns_stations_df_}\DecValTok{200}\NormalTok{)}

\NormalTok{## dataframe}
\NormalTok{model_stations_df <-}\StringTok{ }\NormalTok{swns_stations_df_}\DecValTok{200} \OperatorTok
\StringTok{          }\CommentTok{# Include all study stations}
\StringTok{  }\KeywordTok{filter}\NormalTok{(stationid }\OperatorTok\StringTok{ }\NormalTok{study_stations_list }\OperatorTok{|}
\StringTok{          }\CommentTok{# Included these ext stations}
\StringTok{          }\NormalTok{stationid }\OperatorTok\StringTok{ }\KeywordTok{c}\NormalTok{(}\StringTok{"6354"}\NormalTok{,}\StringTok{"47187"}\NormalTok{,}\StringTok{"6456"}\NormalTok{,}\StringTok{"DNR025"}\NormalTok{,}\StringTok{"6497"}\NormalTok{)) }\OperatorTok\StringTok{ }
\StringTok{          }\CommentTok{# Remove these study stations}
\StringTok{  }\KeywordTok{filter}\NormalTok{((stationid }\OperatorTok\StringTok{ }\KeywordTok{c}\NormalTok{(}\StringTok{"YA4"}\NormalTok{,}\StringTok{"S160"}\NormalTok{,}\StringTok{"AN3"}\NormalTok{,}\StringTok{"CL4"}\NormalTok{,}\StringTok{"CH4"}\NormalTok{,}
                           \StringTok{"KE5"}\NormalTok{,}\StringTok{"WE5"}\NormalTok{,}\StringTok{"WE2"}\NormalTok{,}\StringTok{"LI2"}\NormalTok{,}\StringTok{"BR3"}\NormalTok{,}\StringTok{"MI5"}\NormalTok{)) }\OperatorTok{==}\StringTok{ }\OtherTok{FALSE}\NormalTok{)}
\NormalTok{## spatial points object}
\NormalTok{model_stations_sp <-}\StringTok{ }\NormalTok{model_stations_df }\OperatorTok
\StringTok{                      }\KeywordTok{filter}\NormalTok{(}\OperatorTok{!}\KeywordTok{duplicated}\NormalTok{(stationid))}
\KeywordTok{coordinates}\NormalTok{(model_stations_sp) =}\StringTok{ }\ErrorTok{~}\StringTok{ }\NormalTok{EASTING }\OperatorTok{+}\StringTok{ }\NormalTok{NORTHING}


\CommentTok{# VALIDATION STATIONS}

\NormalTok{## dataframe}
\NormalTok{val_stations_df   <-}\StringTok{ }\NormalTok{swns_stations_df_}\DecValTok{200} \OperatorTok\StringTok{ }
\StringTok{          }\CommentTok{# Include all ext stations}
\StringTok{  }\KeywordTok{filter}\NormalTok{(stationid }\OperatorTok\StringTok{ }\NormalTok{ext_stations_list  }\OperatorTok{|}\StringTok{    }
\StringTok{          }\CommentTok{# Add these study stations}
\StringTok{           }\NormalTok{stationid }\OperatorTok\StringTok{ }\KeywordTok{c}\NormalTok{(}\StringTok{"YA4"}\NormalTok{,}\StringTok{"S160"}\NormalTok{,}\StringTok{"AN3"}\NormalTok{,}\StringTok{"CL4"}\NormalTok{,}\StringTok{"KE5"}\NormalTok{,}\StringTok{"WE5"}\NormalTok{,}\StringTok{"WE2"}\NormalTok{,}\StringTok{"LI2"}\NormalTok{,}\StringTok{"CH4"}\NormalTok{)) }\OperatorTok
\StringTok{          }\CommentTok{# Remove these ext stations}
\StringTok{  }\KeywordTok{filter}\NormalTok{(}\OperatorTok{!}\NormalTok{(stationid }\OperatorTok\StringTok{ }\KeywordTok{c}\NormalTok{(}\StringTok{"6354"}\NormalTok{, }\StringTok{"47187"}\NormalTok{, }\StringTok{"6456"}\NormalTok{,}\StringTok{"27141"}\NormalTok{, }\StringTok{"DLG006"}\NormalTok{, }\StringTok{"DNR024"}\NormalTok{,}
                            \StringTok{"6501"}\NormalTok{, }\StringTok{"DNR025"}\NormalTok{, }\StringTok{"DNR003"}\NormalTok{, }\StringTok{"6497"}\NormalTok{)             }
\NormalTok{           )}
\NormalTok{         )         }

\NormalTok{## spatial points object}
\NormalTok{val_stations_sp   <-}\StringTok{ }\NormalTok{val_stations_df  }\OperatorTok
\StringTok{                      }\KeywordTok{filter}\NormalTok{(}\OperatorTok{!}\KeywordTok{duplicated}\NormalTok{(stationid))}
\KeywordTok{coordinates}\NormalTok{(val_stations_sp) =}\StringTok{ }\ErrorTok{~}\StringTok{ }\NormalTok{EASTING }\OperatorTok{+}\StringTok{ }\NormalTok{NORTHING}

\CommentTok{# Make plot}
\KeywordTok{plot}\NormalTok{(rasters_list[[}\DecValTok{1}\NormalTok{]])}
\KeywordTok{points}\NormalTok{(model_stations_sp, }\DataTypeTok{pch =} \DecValTok{0}\NormalTok{) }\CommentTok{# squares }
\KeywordTok{points}\NormalTok{(val_stations_sp, }\DataTypeTok{pch =} \DecValTok{5}\NormalTok{) }\CommentTok{# diamonds}
\end{Highlighting}
\end{Shaded}

\includegraphics{model_selection_vignette_files/figure-latex/unnamed-chunk-50-1.pdf}

\subsubsection{Generating the test
models}\label{generating-the-test-models}

\begin{itemize}
\tightlist
\item
  More precise predictions made at daily scale. However, n too small for
  amount of variables.
\item
  Issue negated by creating weekly models, with variables that are
  dependent on the day.
\item
  Need to determine whether modelling temperatures maximums and minimums
  separating is more accurate than temperature means.
\item
  With the function validate\_weekly\_GAMs, model statistics and
  residuals are added to the validation stations dataframe.
\item
  In a for loop, the same model was run on the min, max and mean. The
  data frame is bound in the long format to be compatible with plotting
  by ggplot.
\end{itemize}

Visualing the GCV scores strongly suggests that the daily temperature
mean is more appropriate for modelling than the daily minimum or
maximum. The GCV score is consistently below one for the daily
temperature means, whereas daily minimums and maximum hover around 2.

```\{ r eval=FALSE, include=FALSE warning=FALSE\}

ggplot(data = val\_df, mapping = aes(x = yday, y = gcv)) +
geom\_smooth(aes(linetype = model\_opt))

```

A frequency plot (or density, in R), shows that the daily mean
temperature has the most residuals around zero. The steepness of the
slope for daily temperature means also suggests there are less residuals
of a higher values relative to daily minimum and maximum temperatures.

\texttt{\{\ r\ eval=FALSE,\ include=FALSE\ warning=FALSE\}\ ggplot(data\ =\ val\_df,\ mapping\ =\ aes(x\ =\ abs\_resid))\ +\ \ \ geom\_density(aes(colour\ =\ model\_opt))}

-When the residuals are plotted daily for each month, mean temperature
has the lowest smooth line for a majority of the months. - However,
something strange is happening at the end of June. Since all the data
values on this day seem ordinary, something must be happening with the
model.

\begin{itemize}
\tightlist
\item
  So far, GAM has been used without putting limitations on knots.
\item
  It was found that limiting the number of knots worked to reduced the
  erroneous residuals
\end{itemize}

What if i restrict easting and northing to k = 9 also?

The models without knot limitations on E\&N and ptoc = 3 look smaller..
t test to make sure..


\end{document}
